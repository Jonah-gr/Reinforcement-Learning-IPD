\chapter{Fazit}
\section{Zusammenfassung der Ergebnisse}
Zusammenfassend zeigen die Ergebnisse, dass Reinforcement Learning eine effektive Methode ist, um erfolgreiche Strategien für wiederholte 
soziale Dilemmata zu entwickeln. Während der Q-Learning-Agent bereits eine solide Strategie findet, hebt sich der Deep Q-Learning-Agent 
durch seine anpassungsfähige und opportunistische Spielweise hervor. Dies unterstreicht die Stärke von Deep Learning in strategischen 
Entscheidungsszenarien. Trotz einzelner Schwächen zeigt sich, dass lernfähige Agenten klassische, statische Strategien langfristig übertreffen können.

\section{Ausblick}
Dennoch gibt es einige Herausforderungen und 
offene Fragen, die in zukünftigen Arbeiten weiter untersucht werden sollten.

Ein interessantes Thema für zukünftige Forschungen könnte die Anwendung von Langzeitgedächtnisstrategien sein, 
wie sie in rekurrenten neuronalen Netzwerken (RNNs) oder Long Short-Term Memory-Netzwerken (LSTMs) zu finden sind. 
Diese könnten es den Agenten ermöglichen, frühere Interaktionen und Muster in den Entscheidungen des Gegners besser zu berücksichtigen 
und ihre Strategie entsprechend anzupassen. Auch der Einfluss von ``Vergessen'' in diesen Netzwerken könnte interessante neue 
Erkenntnisse liefern, da in manchen Situationen das Behalten von Informationen über sehr lange Zeiträume kontraproduktiv sein könnte.

Die Untersuchung alternativer Netzwerk-Architekturen und Hyperparametern ist ein weiteres Feld für zukünftige Arbeiten. 
Dies würde es ermöglichen, das Potenzial der Agenten weiter zu maximieren.

Schließlich ist auch die Frage nach der Erklärbarkeit der Entscheidungen von Deep Q-Learning-Agenten ein spannendes Thema. 
Während diese Agenten in Bezug auf die Performance herausragend sind, bleibt es oft unklar, warum sie bestimmte Entscheidungen treffen. 
In sicherheitskritischen oder ethisch sensiblen Anwendungen könnte es entscheidend sein, diese Entscheidungen nachvollziehbar 
und transparent zu machen. Ansätze wie Explainable AI könnten hier dazu beitragen, die Zugänglichkeit und das 
Vertrauen in die Modelle zu verbessern.

Insgesamt zeigt die Forschung, dass Reinforcement Learning und Deep Learning das Potenzial haben, die Entscheidungsfindung in sozialen Dilemmata zu revolutionieren. Dennoch bleibt noch viel Raum für Verbesserung und Erweiterung, um die Agenten noch robuster und anpassungsfähiger zu machen, insbesondere in realeren Szenarien, in denen Unsicherheit und Komplexität eine größere Rolle spielen.
